\section{Protocolo IP} %

\lipsum[3-9][4-8].

El protocolo IP (Internet Protocol) Junto con el protocolo TCP (Transmission Control Protocol) son los dos protocolos más importantes en el Internet.

IP determinará el formato de los paquetes de datos que se envían y reciben a través de la red. TCP se encargará de la transmisión de los datos.

Existen dos versiones de IP, la versión 4 (IPv4) y la versión 6 (IPv6). La versión 4 es la más utilizada en la actualidad, pero se está migrando a la versión 6 debido a la falta de direcciones IP disponibles en la versión 4.

\section{Protocolo UDP} %



\section{VPN} %
\section{Wireguard} %

\section{NAT} %
\section{IP Masquerade} %

\section{Tablas de routeo}  %https://docs.aws.amazon.com/vpc/latest/userguide/VPC_Route_Tables.html


\section{IPTables} %
IPTables es una herramienta de filtrado de paquetes de red que se utiliza en sistemas operativos basados en Linux.

Es una utilidad de línea de comandos que permite configurar el firewall de Linux. 

IPTables es una herramienta muy poderosa que permite configurar reglas de filtrado de paquetes, redireccionamiento de paquetes, traducción de direcciones de red, etc.


\section{Relay network}%

%\section{Firewall} %
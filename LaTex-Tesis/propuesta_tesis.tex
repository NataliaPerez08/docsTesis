\documentclass[letterpaper,12pt,oneside]{book}
\usepackage[top=1in, left=1.25in, right=1.25in, bottom=1in]{geometry}
\usepackage[utf8]{inputenc}
\usepackage{graphicx}

\usepackage[spanish]{babel}

%\title{Propuesta de Tesis: LinkGuard, MeshGuard, WireFleet} 2962.95

% Título tentativo
\title{Orquestrador para Wireguard minimalista: LinkGuard}

\begin{document}

\maketitle

%Propuesta de tema de Tesis (Título tentativo, justificación, objetivos, índice tentativo, bibliografía básica en orden alfabético), firmada de forma autógrafa por el estudiante y por el tutor.

\section{Introducción}

En la actualidad, y desde hace tiempo, las empresas al tener un presencia global con multiples sucursales y empleados remotos requieren establecer conexiones seguras entre sus dispositivos. Para esto hay dos opciones: una línea de alquiler privada y dedicada o compartir una parte del ancho de banda con una línea existente, como Internet. La segunda opción es más económica y flexible, pero menos segura. De ahí surgen las redes privadas virtuales (VPN) que permiten establecer un túnel seguro dentro de un red publica tal como si estuvieran conectados en una red local. Adicionalmente, usuarios finales han encontrado en las VPN una forma de proteger su información y privacidad, ademas de acceder a contenido restringido geograficamente.

Diferentes protocolos de VPN han sido desarrollados para responder estos requerimientos, cada uno con sus propias características y funcionalidades, entre ellos OpenVPN y IPsec. Sin embargo, estos protocolos presentan problemas de seguridad, complejidad y rendimiento. Por ejemplo, OpenVPN es un protocolo de VPN de código abierto que utiliza el protocolo SSL/TLS para cifrar el tráfico de red, pero es lento y complejo de configurar. Por otro lado, IPsec es un protocolo de VPN que utiliza el protocolo IKEv2 para establecer un túnel seguro, pero es difícil de configurar y no es compatible con todos los dispositivos.

En respuesta a estos problemas, se ha desarrollado un protocolo de VPN llamado Wireguard que es más simple, más rápido y más seguro que otros protocolos de VPN. Wireguard es más simple que otros protocolos de VPN porque utiliza un enfoque basado en claves públicas para establecer un túnel seguro, en lugar de utilizar certificados SSL/TLS como OpenVPN. Wireguard es más rápido que otros protocolos de VPN porque utiliza un enfoque basado en el kernel para cifrar y descifrar el tráfico de red, en lugar de utilizar un enfoque basado en el usuario como OpenVPN. Wireguard es más seguro que otros protocolos de VPN porque utiliza un enfoque basado en claves públicas para establecer un túnel seguro, en lugar de utilizar un enfoque basado en contraseñas como IPsec.

Si bien Wireguard es un protocolo de VPN simple la mayor de sus desventajas es la configuración de los pares dentro de la red. Por ejemplo, si se desea configurar una red privada virtual con 10 pares, se debe configurar manualmente cada par con la dirección IP y la clave pública de cada par. Esto puede ser un proceso tedioso y propenso a errores, especialmente si se desea configurar una red privada virtual con un gran número de pares.

Una solución para esta dificultad es propuesa por Tailscale, el cual es un servicio VPN que hace que sus dispositivos y aplicaciones sean accesibles en cualquier parte del mundo, de forma segura y sin esfuerzo. Este software actúa en combinación con el kernel para establecer una comunicación VPN peer-to-peer o retransmitida con otros clientes utilizando el protocolo WireGuard. Tailscale puede abrir una conexión directa con el peer utilizando técnicas de NAT traversal como STUN o solicitar el reenvío de puertos a través de UPnP IGD, NAT-PMP o PCP. [7] Si el software no consigue establecer una comunicación directa, recurre al protocolo DERP (Designated Encrypted Relay for Packets) proporcionados por la empresa. [6] Las direcciones IPv4 asignadas a los clientes se encuentran en el espacio reservado NAT de nivel operador. Esto se eligió para evitar interferencias con las redes existentes[9] ya que el enrutamiento de tráfico a redes detrás del cliente es posible.

Tailscale es una gran solución para la configuración de pares en Wireguard, pero no es una solución de código abierto. Al ofrecer mas funcionalidades que las necesarias para la configuración de pares en Wireguard, Tailscale puede ser una solución costosa para empresas pequeñas o individuos que solo requieren configurar pares en Wireguard. Además, Tailscale no permite a los usuarios tener control total sobre la configuración de pares en Wireguard, ya que la configuración de pares en Wireguard se realiza a través de la interfaz de usuario de Tailscale y no a través de la línea de comandos. Y finalmente, Tailscale aumenta la superficie de ataque de la red privada virtual, ya que no solicita la autenticación al usar un servicio crítico como SSH entre los pares.

Existe una alternativa de código abierto del servidor de control de Tailscale llamado Headscale. El objetivo de Headscale es proporcionar a un servidor de código abierto que puedan utilizar para proyectos y laboratorios. Implementa un alcance estrecho, una sola Tailnet, adecuada para un uso personal o una pequeña organización de código abierto. [10]



Por lo que en este trabajo de tesis se propone el desarrollo de un prototipo open-source de un sistema de control de configuración de pares minimalista (adherido al principio de UNIX de que un componenete solo hace una y solo una cosa muy bien) que se basará en el protocolo Wireguard (inspirado en Tailscale) %CORREGIR y permitirá a los usuarios automatizar la configuración de pares. 
%ya que la configuración de pares en Wireguard se realizará a través de la línea de comandos y no a través de una interfaz de usuario. 
% Una herramienta por funcion

% Alcance limitado, *prototipo*
Además se espera que este prototipo de sistema de control de configuración de pares en Wireguard minimalista reducirá la superficie de ataque de la red privada virtual, ya que solicitará la autenticación al usar un servicio crítico como SSH entre los pares.

\section{Justificación}
    
Ante los requerimientos de usuarios finales y empresas de establecer conexiones seguras entre sus dispositivos surgen las VPN. Surgen diferentes protocolos entre Wireguard, el cual destaca por simple, rápido y seguro. Sin embargo la configuración de pares puede ser un proceso tedioso y propenso a errores.

La necesidad de proteger la información y la privacidad de los usuarios no deberia estar limitada por el tiempo y la complejidad de configuración de los protocolos de VPN.

Por l que 

El desarrollo de un sistema de control de configuración de pares en una red privada virtual que utiliza el protocolo Wireguard permitirá a los usuarios configurar de forma sencilla y segura los pares de la red privada virtual.

Ademas de permitir la comunicacion segura entre los pares de la red privada virtual que pueden estar detras de firewalls y NATs.

Tailscale.



\section{Objetivos}

El desarrollo de la tesis tiene como objetivo principal el desarrollo de un sistema de orquestramiento de pares dentro de una red privada virtual que utiliza el protocolo Wireguard.
Cooparativa con TailScale

\section{Índice Tentativo}
\begin{flushleft}
    \begin{tabbing}
    \hspace{4cm} \= \kill
    1. Introducción \> \dotfill  \\
    \hspace{0.5cm}1.1 VPN \> \dotfill  \\
    \hspace{0.5cm}1.2 Wireguard \> \dotfill  \\
    \hspace{0.5cm}1.3 Tailscale \> \dotfill  \\
    \hspace{0.5cm}1.4 NAT \> \dotfill  \\
    \hspace{0.5cm}1.1 Firewall \> \dotfill  \\

    2. Desarrollo \> \dotfill  \\

    \hspace{0.5cm}2.0 Objetivos del programa \> \dotfill  \\
    \hspace{0.5cm}2.1 Funcionalidad del orquestrador \> \dotfill  \\
    \hspace{0.5cm}2.2 Descripcion del orquestrador \> \dotfill  \\
    \hspace{0.5cm}2.3 Componentes del orquestrador \> \dotfill  \\
    \hspace{0.5cm}2.4 Flujo del programa \> \dotfill  \\
    \hspace{0.5cm}2.5 Casos de uso \> \dotfill  \\
    
    3. Pruebas y evaluación de resultados \> \dotfill \\
    \hspace{0.5cm}3.1 Metodología de evaluación \> \dotfill 8 \\
    \hspace{0.5cm}3.2 Resultados de la evaluación \> \dotfill 12 \\
    4. Conclusiones y trabajo a futuro \> \dotfill  \\
    5. Bibliografía \> \dotfill 22 \\
    \end{tabbing}
    \end{flushleft}

\section{Bibliografía Básica}
[1] Kurose, J. F., y Ross, K. W. (2017). Computer networking: a top-down approach,
Pearson, 7th edition.

[2] Narayan, S., Williams, C. J., Hart, D. K., y Qualtrough, M. W. (2015). Network performance comparison of VPN protocols on wired and wireless networks. 2015 International Conference on Computer Communication and Informatics (ICCCI). doi:10.1109/iccci.2015.7218077

[3]  Abdulazeez, A., Salim, B., Zeebaree, D. y Doghramachi, D. (2020). Comparison of VPN Protocols at Network Layer Focusing on Wire Guard Protocol. International Association of Online Engineering. Retrieved September 26, 2024 from https://www.learntechlib.org/p/218341/. 

[4] WireGuard, WireGuard: fast, modern, secure VPN tunnel, https://www.wireguard.com/, 2024.

[5] What is Tailscale?, https://tailscale.com/kb/1151/what-is-tailscale/, 2024.

[6] Terminology and concepts, https://tailscale.com/kb/1155/terminology-and-concepts\#relay, 2024

[7] Troubleshooting device connectivity, https://tailscale.com/kb/1411/device-connectivity\#nat-types, 2024

[8] How NAT traversal works, https://tailscale.com/blog/how-nat-traversal-works, 2024

[9] IP pool, https://tailscale.com/kb/1304/ip-pool, 2024

[10] Headscale, 

[11] Linux Documentation Project, Linux Advanced Routing y Traffic Control HOW-
TO, https://tldp.org/HOWTO/Adv-Routing-HOWTO/index.html, 2021.  

[12] Bautts, M., y Dawson, M. (2000). Linux Network Administrator’s Guide, O’Reilly
Media, 3rd edition.

[13] Bautts, M., y Dawson, M. (2000). Linux IP Masquerade HOWTO, https://
tldp.org/HOWTO/IP-Masquerade-HOWTO/index.html, 2021


\end{document}








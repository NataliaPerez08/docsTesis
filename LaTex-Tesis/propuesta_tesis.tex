\documentclass{article}
\usepackage[utf8]{inputenc}
\usepackage{graphicx}

\usepackage[o]{babel}

%\title{Propuesta de Tesis: LinkGuard, MeshGuard, WireFleet}

% Título tentativo
\title{Orquestrador para Wireguard minimalista: LinkGuard, MeshGuard, WireFleet}

\begin{document}

\maketitle

%Propuesta de tema de Tesis (Título tentativo, justificación, objetivos, índice tentativo, bibliografía básica en orden alfabético), firmada de forma autógrafa por el estudiante y por el tutor.

\section{Introducción}

En la actualidad tanto el usuario común como las empresas requieren un método de comunicación que permita mantener confidencialidad y privacidad entre sus dispositivos. 



Por tanto diferentes protocolos de VPN han sido desarrollados para satisfacer las necesidades de los usuarios entre ellos destaca Wireguard por ser un protocolo de VPN de código abierto, de alto rendimiento, más simple y más rápido que otros protocolos de VPN, como OpenVPN y IPsec.


La mayor dificultad al configurar un VPN utilizando el protocolo Wireguard es la configuración de los pares dentro de la red. Por ejemplo 

Por lo que se propone el desarrollo de un sistema de control de configuración de pares en Wireguard.

\section{Justificación}
    
La necesidad de proteger la información y la privacidad de los usuarios no deberia estar limitada por el tiempo y la complejidad de configuración de los protocolos de VPN. 



El desarrollo de un sistema de control de configuración de pares en una red privada virtual que utiliza el protocolo Wireguard permitirá a los usuarios configurar de forma sencilla y segura los pares de la red privada virtual.

Ademas de permitir la comunicacion segura entre los pares de la red privada virtual que pueden estar detras de firewalls y NATs.

\section{Objetivos}

El desarrollo de la tesis tiene como objetivo principal el desarrollo de un sistema de orquestramiento de pares dentro de una red privada virtual que utiliza el protocolo Wireguard.
Cooparativa con TailScale

\section{Índice Tentativo}

1. Introduccion:
    1.1 VPN
    1.2 Wireguard
    1.3 TailScale
    1.3 NATs
    1.4 firewalls

2. Desarrollo
    2.0 Objetivos del programa
    2.1 Funcionalidad del orquestrador
    2.2 Descripcion del orquestrador
    2.2.1 Componentes del orquestrador
    2.2.2 Flujo del programa
    2.3 Casos de uso
3. Pruebas y Evaluación de resultados 
    3.0 Metodología de evaluación
    3.1 Resultados de la evaluación
4. Conclusiones y trabajo futuro 
5. Bibliografía 

\section{Bibliografía Básica}
[1] Kurose, J. F., & Ross, K. W. (2017). Computer networking: a top-down approach,
Pearson, 7th edition.

[2] WireGuard, WireGuard: fast, modern, secure VPN tunnel, https://www.wireguard.com/, 2024.

[3] What is Tailscale?, https://tailscale.com/kb/1151/what-is-tailscale/, 2024.

[4] Linux Documentation Project, Linux Advanced Routing & Traffic Control HOW-
TO, https://tldp.org/HOWTO/Adv-Routing-HOWTO/index.html, 2021.

[3] Bautts, M., & Dawson, M. (2000). Linux Network Administrator’s Guide, O’Reilly
Media, 3rd edition.

[4] Bautts, M., & Dawson, M. (2000). Linux IP Masquerade HOWTO, https://
tldp.org/HOWTO/IP-Masquerade-HOWTO/index.html, 2021


\end{document}








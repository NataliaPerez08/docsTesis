\documentclass{article}
\usepackage[utf8]{inputenc}
\usepackage{graphicx}

\usepackage[o]{babel}

\title{Propuesta de Tesis}

\begin{document}

\maketitle

Propuesta de tema de Tesis (Título tentativo, justificación, objetivos, índice tentativo, bibliografía básica en orden alfabético), firmada de forma autógrafa por el estudiante y por el tutor.
\section{Introducción}
Diferentes protocolos de VPN han sido desarrollados para satisfacer las necesidades de los usuarios, como la privacidad y la seguridad de la información.
En el presente trabajo nos enfocamos en Wireguard, un protocolo de VPN de código abierto, de alto rendimiento, más simple y más rápido que otros protocolos de VPN, como OpenVPN y IPsec.

Uno de las dificultades de los protocolos de VPN es la configuración de los pares dentro de la red privada virtual, por lo que se propone el desarrollo de un sistema de control de configuración de pares en una red privada virtual que utiliza el protocolo Wireguard.

\section{Justificación}

La necesidad de proteger la información y la privacidad de los usuarios no deberia estar limitada por el tiempo y la complejidad de configuración de los protocolos de VPN. 

El desarrollo de un sistema de control de configuración de pares en una red privada virtual que utiliza el protocolo Wireguard permitirá a los usuarios configurar de forma sencilla y segura los pares de la red privada virtual.

Ademas de permitir la comunicacion segura entre los pares de la red privada virtual que pueden estar detras de firewalls y NATs.

\section{Objetivos}

El desarrollo de la tesis tiene como objetivo principal el desarrollo de un sistema de orquestramiento de pares dentro de una red privada virtual que utiliza el protocolo Wireguard.


\section{Índice Tentativo}

\section{Bibliografía Básica}

\end{document}








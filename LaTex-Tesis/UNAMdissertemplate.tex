\documentclass[letterpaper,12pt,oneside]{book}
\usepackage[top=1in, left=1.25in, right=1.25in, bottom=1in]{geometry}
\usepackage{bachelorstitlepageUNAM}

%%%%%%%%%%%%%%%%%%%%%%%%%%%%%
% Comparto una plantilla para la PORTADA que us\'e en mi t\'esis
% basada en el dise\~no gen\'erico que se usa en la Facultad de Ciencias
% Para usarlo \'unicamente aseg\'urate de tener la l\'inea
% \usepackage{bachelorstitlepageUNAM} y el archivo bachelorstitlepageUNAM.sty en el mismo directorio de trabajo.
% y los campos (sin signo %) :
%\author{Nombre del Alumno}
%\title{T\'itulo de la tesis}
%\faculty{Facultad}
%\degree{Grado que obtienes}
%\supervisor{ Tutor}
%\cityandyear{Ciudad y anio}
%\logouni{nombredelescudodelaunamsinespacios}
%\logofac{NombreDeLaImagenDelEscudodeTuFacultadSinEspacios}
% Para sugerencias y comentarios: DM en twitter.com/sglvgdor
% Subir\'e mas adelante la plantilla para maestr\'ia
%%%%%%%%%%%%%%%%%%%%%%%%%%%%%

\author{Pérez Romero Natalia Abigail}
\title{ ...}
\faculty{Facultad de Ciencias}
\degree{Licenciatura en Ciencias de la Computación}
\supervisor{Dr. José David Flores Peñaloza}
\cityandyear{Ciudad Universitaria, Cd. Mx., 2023}
\logouni{Escudo-UNAM}
\logofac{Escudo-FCIENCIAS}

\usepackage[T1]{fontenc}
\usepackage[utf8]{inputenc}
\usepackage[spanish,es-nodecimaldot,es-tabla]{babel}
\usepackage{graphicx}
\usepackage{tikz}

\graphicspath{{./figs/}}
\usepackage{setspace}
%\usepackage[round]{natbib}

\usepackage{lipsum}

\begin{document}
\frontmatter
\maketitle

\chapter*{}
\begin{flushright}%
  \emph{Dedicatoria ...}
  \thispagestyle{empty}
\end{flushright}

\chapter{Agradecimientos}
\spacing{1.5}%\doublespacing
%\chapter{Resumen}
%\chapter{Abstract}

\tableofcontents
%\listoffigures

\chapter{Prefacio}

\section*{Contenido de la tesis}
\section*{Panorama general}
\section*{Objetivo}
    
\mainmatter

\chapter{Introducción} %

\section{Protocolo IP} %

\lipsum[3-9][4-8].

El protocolo IP (Internet Protocol) Junto con el protocolo TCP (Transmission Control Protocol) son los dos protocolos más importantes en el Internet.

IP determinará el formato de los paquetes de datos que se envían y reciben a través de la red. TCP se encargará de la transmisión de los datos.

Existen dos versiones de IP, la versión 4 (IPv4) y la versión 6 (IPv6). La versión 4 es la más utilizada en la actualidad, pero se está migrando a la versión 6 debido a la falta de direcciones IP disponibles en la versión 4.

\section{Protocolo UDP} %



\section{VPN} %
\section{Wireguard} %

\section{NAT} %
\section{IP Masquerade} %

\section{Tablas de routeo}  %https://docs.aws.amazon.com/vpc/latest/userguide/VPC_Route_Tables.html


\section{IPTables} %
IPTables es una herramienta de filtrado de paquetes de red que se utiliza en sistemas operativos basados en Linux.

Es una utilidad de línea de comandos que permite configurar el firewall de Linux. 

IPTables es una herramienta muy poderosa que permite configurar reglas de filtrado de paquetes, redireccionamiento de paquetes, traducción de direcciones de red, etc.


\section{Relay network}%

%\section{Firewall} %

  

\section{Sección}
    
    \subsection{Subsecci\'on uno}
    \begin{equation}
    	x_n = c \, x_n(1 - x_n)
    	\label{Ec:logis}
    \end{equation}

\chapter{Resultados}  %

\chapter{Conclusiones}  %

\bibliographystyle{humannat}
%s
\bibliography{references}

\backmatter@sglvgdor

% Referencias
%https://www.overleaf.com/learn/latex/Bibliography_management_with_bibtex

\begin{thebibliography}{9}
\bibitem{computerNetworking}
  Kurose, J. F., \& Ross, K. W. (2017).
  \emph{Computer networking: a top-down approach},
  Pearson,
  7th edition,
  2017.
\end{thebibliography}

\end{document}
\documentclass[letterpaper,12pt,oneside]{book}
\usepackage[top=1in, left=1.25in, right=1.25in, bottom=1in]{geometry}
\usepackage{bachelorstitlepageUNAM}

%%%%%%%%%%%%%%%%%%%%%%%%%%%%%
% Comparto una plantilla para la PORTADA que us\'e en mi t\'esis
% basada en el dise\~no gen\'erico que se usa en la Facultad de Ciencias
% Para usarlo \'unicamente aseg\'urate de tener la l\'inea
% \usepackage{bachelorstitlepageUNAM} y el archivo bachelorstitlepageUNAM.sty en el mismo directorio de trabajo.
% y los campos (sin signo %) :
%\author{Nombre del Alumno}
%\title{T\'itulo de la tesis}
%\faculty{Facultad}
%\degree{Grado que obtienes}
%\supervisor{ Tutor}
%\cityandyear{Ciudad y anio}
%\logouni{nombredelescudodelaunamsinespacios}
%\logofac{NombreDeLaImagenDelEscudodeTuFacultadSinEspacios}
% Para sugerencias y comentarios: DM en twitter.com/sglvgdor
% Subir\'e mas adelante la plantilla para maestr\'ia
%%%%%%%%%%%%%%%%%%%%%%%%%%%%%

\author{Pérez Romero Natalia Abigail}
\title{ ...}
\faculty{Facultad de Ciencias}
\degree{Licenciatura en Ciencias de la Computación}
\supervisor{Dr. José David Flores Peñaloza}
\cityandyear{Ciudad Universitaria, Cd. Mx., 2023}
\logouni{Escudo-UNAM}
\logofac{Escudo-FCIENCIAS}

\usepackage[T1]{fontenc}
\usepackage[utf8]{inputenc}
\usepackage[spanish,es-nodecimaldot,es-tabla]{babel}
\usepackage{graphicx}
\usepackage{tikz}
\usepackage{minted}
% use hyperref to make the table of contents clickable
\usepackage{hyperref}

\graphicspath{{"imagenes/"}}
\usepackage{setspace}
%\usepackage[round]{natbib}

\usepackage{lipsum}

\begin{document}
\frontmatter
\maketitle


\mainmatter

\chapter{title}

\chapter{Introducción} %

Las redes virtuales privadas (VPN) han sido de gran utilidad en el internet moderno desde necesidades empresariales como la transmisión de datos entre diferentes oficinas, como uso personal para proteger la privacidad de los usuarios. 

Existen protocolos de VPN como QUIC, OpenVPN, IPsec, Wireguard, entre otros. 
En el presente trabajo nos enfocamos en Wireguard, un protocolo de VPN de código abierto, de alto rendimiento, más simple y más rápido que otros protocolos de VPN, como OpenVPN y IPsec.



%\section{Protocolo IP} %

\lipsum[3-9][4-8].

El protocolo IP (Internet Protocol) Junto con el protocolo TCP (Transmission Control Protocol) son los dos protocolos más importantes en el Internet.

IP determinará el formato de los paquetes de datos que se envían y reciben a través de la red. TCP se encargará de la transmisión de los datos.

Existen dos versiones de IP, la versión 4 (IPv4) y la versión 6 (IPv6). La versión 4 es la más utilizada en la actualidad, pero se está migrando a la versión 6 debido a la falta de direcciones IP disponibles en la versión 4.

\section{Protocolo UDP} %



\section{VPN} %
\section{Wireguard} %

\section{NAT} %
\section{IP Masquerade} %

\section{Tablas de routeo}  %https://docs.aws.amazon.com/vpc/latest/userguide/VPC_Route_Tables.html


\section{IPTables} %
IPTables es una herramienta de filtrado de paquetes de red que se utiliza en sistemas operativos basados en Linux.

Es una utilidad de línea de comandos que permite configurar el firewall de Linux. 

IPTables es una herramienta muy poderosa que permite configurar reglas de filtrado de paquetes, redireccionamiento de paquetes, traducción de direcciones de red, etc.


\section{Relay network}%

%\section{Firewall} %

\chapter{Desarrollo}

\section{Casos de uso}
\subsection{Identificación del usuario}

En esta primera version no nos preocuparemos de que la información del usuario se transmita de forma segura, por lo que el usuario deberá ingresar su nombre y contraseña en texto plano. Y se enviará al servidor para su verificación.

\begin{figure}
    \centering
    \includegraphics[width=\textwidth]{login-user.png}
    \caption{Pantalla de inicio de sesión}
\end{figure}

\subsubsection{Registro de usuario}

De igual forma el registro de usuario se hará en texto plano, y se enviará al servidor para su verificación.

\begin{figure}
    \centering
    \includegraphics[width=\textwidth]{register-user.png}
    \caption{Pantalla de registro de usuario}
\end{figure}

\subsection{Registro de usuario}

\lipsum[1-2]
\newpage
\input{casos-conexion.tex}

\newpage
\section{Diagrama de clases}
\input{diagrama-clases.tex}
%\lipsum[1-2]

%\chapter{Resultados}  %

%\chapter{Conclusiones}  %

\bibliographystyle{humannat}
%\bibliography{references}

%\backmatter@sglvgdor

% Referencias
%https://www.overleaf.com/learn/latex/Bibliography_management_with_bibtex

\begin{thebibliography}{9}
\bibitem{computerNetworking}
  Kurose, J. F., \& Ross, K. W. (2017).
  \emph{Computer networking: a top-down approach},
  Pearson,
  7th edition.

\bibitem{wireguard}
    WireGuard,
    \emph{WireGuard: fast, modern, secure VPN tunnel},
    \url{https://www.wireguard.com/},
    2021.

\bibitem{linuxNetworkingGuide}
    Linux Documentation Project,
    \emph{Linux Advanced Routing \& Traffic Control HOWTO},
    \url{https://tldp.org/HOWTO/Adv-Routing-HOWTO/index.html},
    2021.

\bibitem{networkAdministartionGuide}
    Bautts, M., \& Dawson, M. (2000).
    \emph{Linux Network Administrator's Guide},
    O'Reilly Media,
    3rd edition.

% https://tldp.org/HOWTO/IP-Masquerade-HOWTO/kernel-2.4.x-requirements.html

\bibitem{ipMasquerade}
    Bautts, M., \& Dawson, M. (2000).
    \emph{Linux IP Masquerade HOWTO},
    \url{https://tldp.org/HOWTO/IP-Masquerade-HOWTO/index.html},
    2021.

\end{thebibliography}

\end{document}
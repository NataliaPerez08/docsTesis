\documentclass[letterpaper,12pt,oneside]{book}
\usepackage[top=1in, left=1.25in, right=1.25in, bottom=1in]{geometry}
\usepackage{bachelorstitlepageUNAM}

%%%%%%%%%%%%%%%%%%%%%%%%%%%%%
% Comparto una plantilla para la PORTADA que us\'e en mi t\'esis
% basada en el dise\~no gen\'erico que se usa en la Facultad de Ciencias
% Para usarlo \'unicamente aseg\'urate de tener la l\'inea
% \usepackage{bachelorstitlepageUNAM} y el archivo bachelorstitlepageUNAM.sty en el mismo directorio de trabajo.
% y los campos (sin signo %) :
%\author{Nombre del Alumno}
%\title{T\'itulo de la tesis}
%\faculty{Facultad}
%\degree{Grado que obtienes}
%\supervisor{ Tutor}
%\cityandyear{Ciudad y anio}
%\logouni{nombredelescudodelaunamsinespacios}
%\logofac{NombreDeLaImagenDelEscudodeTuFacultadSinEspacios}
% Para sugerencias y comentarios: DM en twitter.com/sglvgdor
% Subir\'e mas adelante la plantilla para maestr\'ia
%%%%%%%%%%%%%%%%%%%%%%%%%%%%%

\author{Pérez Romero Natalia Abigail}
\title{ ...}
\faculty{Facultad de Ciencias}
\degree{Licenciatura en Ciencias de la Computación}
\supervisor{Dr. José David Flores Peñaloza}
\cityandyear{Ciudad Universitaria, Cd. Mx., 2023}
\logouni{Escudo-UNAM}
\logofac{Escudo-FCIENCIAS}

\usepackage[T1]{fontenc}
\usepackage[utf8]{inputenc}
\usepackage[spanish,es-nodecimaldot,es-tabla]{babel}
\usepackage{graphicx}
\usepackage{tikz}

\graphicspath{{./figs/}}
\usepackage{setspace}
%\usepackage[round]{natbib}

\begin{document}
\frontmatter
\maketitle

\chapter*{}
\begin{flushright}%
  \emph{Dedicatoria ...}
  \thispagestyle{empty}
\end{flushright}

\chapter{Agradecimientos}
\spacing{1.5}%\doublespacing
%\chapter{Resumen}
%\chapter{Abstract}

\tableofcontents
%\listoffigures

\chapter{Prefacio}

\section*{Contenido de la tesis}
\section*{Panorama general}
\section*{Objetivo}
    
\mainmatter

\chapter{Introducción} %

\section{Protocolo IP} %
\section{Protocolo UDP} %
\section{VPN} %
\section{Wireguard} %

\section{NAT} %
\section{IP Masquerade} %
%\section{Firewall} %

  

\section{Sección}
    
    \subsection{Subsecci\'on uno}
    \begin{equation}
    	x_n = c \, x_n(1 - x_n)
    	\label{Ec:logis}
    \end{equation}

\chapter{Resultados}  %

\chapter{Conclusiones}  %

%\bibliographystyle{humannat}
%\bibliography{references}

\backmatter%@sglvgdor
\end{document}
\documentclass[letterpaper,12pt,oneside]{book}
\usepackage[top=1in, left=1.25in, right=1.25in, bottom=1in]{geometry}
\usepackage{bachelorstitlepageUNAM}

%%%%%%%%%%%%%%%%%%%%%%%%%%%%%
% Comparto una plantilla para la PORTADA que us\'e en mi t\'esis
% basada en el dise\~no gen\'erico que se usa en la Facultad de Ciencias
% Para usarlo \'unicamente aseg\'urate de tener la l\'inea
% \usepackage{bachelorstitlepageUNAM} y el archivo bachelorstitlepageUNAM.sty en el mismo directorio de trabajo.
% y los campos (sin signo %) :
%\author{Nombre del Alumno}
%\title{T\'itulo de la tesis}
%\faculty{Facultad}
%\degree{Grado que obtienes}
%\supervisor{ Tutor}
%\cityandyear{Ciudad y anio}
%\logouni{nombredelescudodelaunamsinespacios}
%\logofac{NombreDeLaImagenDelEscudodeTuFacultadSinEspacios}
% Para sugerencias y comentarios: DM en twitter.com/sglvgdor
% Subir\'e mas adelante la plantilla para maestr\'ia
%%%%%%%%%%%%%%%%%%%%%%%%%%%%%

\author{Pérez Romero Natalia Abigail}
\title{ ...}
\faculty{Facultad de Ciencias}
\degree{Licenciatura en Ciencias de la Computación}
\supervisor{Dr. José David Flores Peñaloza}
\cityandyear{Ciudad Universitaria, Cd. Mx., 2023}
\logouni{Escudo-UNAM}
\logofac{Escudo-FCIENCIAS}

\usepackage[T1]{fontenc}
\usepackage[utf8]{inputenc}
\usepackage[spanish,es-nodecimaldot,es-tabla]{babel}
\usepackage{graphicx}
\usepackage{tikz}
\usepackage{minted}
% use hyperref to make the table of contents clickable
\usepackage{hyperref}

\graphicspath{{"imagenes/"}}
\usepackage{setspace}
%\usepackage[round]{natbib}

\usepackage{lipsum}

\begin{document}
\frontmatter
\maketitle


\mainmatter

\chapter{title}

% Crear indice
\tableofcontents

\chapter{Introducción} %

% Introduccion, justificacion y objetivos
\input{introduccion.tex}

% Definiciones
%
\section{VPN} %
Las VPN (Virtual Private Netwo  k) son redes privadas virtuales que permiten a los usuarios conectarse a una red privada a través de una red pública, como Internet. Las VPN se utilizan para proteger la privacidad y la seguridad de la información transmitida a través de la red.



\section{WireGuard}


\section{Tailscale}



\section{NAT}


\section{Firewall}

\chapter{Desarrollo}
%2. Desarrollo \> \dotfill  \\

%\hspace{0.5cm}2.0 Objetivos del programa \> \dotfill  \\
%\hspace{0.5cm}2.1 Funcionalidad del orquestrador \> \dotfill  \\
%\hspace{0.5cm}2.2 Descripcion del orquestrador \> \dotfill  \\
%\hspace{0.5cm}2.3 Componentes del orquestrador \> \dotfill  \\
%\hspace{0.5cm}2.4 Flujo del programa \> \dotfill  \\
%\hspace{0.5cm}2.5 Casos de uso \> \dotfill  \\

\input{objetivosPrograma.tex}

\section{Casos de uso}
\subsection{Identificación del usuario}

En esta primera version no nos preocuparemos de que la información del usuario se transmita de forma segura, por lo que el usuario deberá ingresar su nombre y contraseña en texto plano. Y se enviará al servidor para su verificación.

\begin{figure}
    \centering
    \includegraphics[width=\textwidth]{login-user.png}
    \caption{Pantalla de inicio de sesión}
\end{figure}

\subsubsection{Registro de usuario}

De igual forma el registro de usuario se hará en texto plano, y se enviará al servidor para su verificación.

\begin{figure}
    \centering
    \includegraphics[width=\textwidth]{register-user.png}
    \caption{Pantalla de registro de usuario}
\end{figure}

\subsection{Registro de usuario}

\lipsum[1-2]
\newpage
\input{casos-conexion.tex}

\newpage
\section{Diagrama de clases}
\input{diagrama-clases.tex}
%\lipsum[1-2]

\chapter{Resultados}  %

\chapter{Conclusiones}  %

\bibliographystyle{humannat}
%\bibliography{references}

%\backmatter@sglvgdor

% Referencias
%https://www.overleaf.com/learn/latex/Bibliography_management_with_bibtex

\begin{thebibliography}{9}
\bibitem{computerNetworking}
  Kurose, J. F., \& Ross, K. W. (2017).
  \emph{Computer networking: a top-down approach},
  Pearson,
  7th edition.

\bibitem{wireguard}
    WireGuard,
    \emph{WireGuard: fast, modern, secure VPN tunnel},
    \url{https://www.wireguard.com/},
    2021.

\bibitem{linuxNetworkingGuide}
    Linux Documentation Project,
    \emph{Linux Advanced Routing \& Traffic Control HOWTO},
    \url{https://tldp.org/HOWTO/Adv-Routing-HOWTO/index.html},
    2021.

\bibitem{networkAdministartionGuide}
    Bautts, M., \& Dawson, M. (2000).
    \emph{Linux Network Administrator's Guide},
    O'Reilly Media,
    3rd edition.

% https://tldp.org/HOWTO/IP-Masquerade-HOWTO/kernel-2.4.x-requirements.html

\bibitem{ipMasquerade}
    Bautts, M., \& Dawson, M. (2000).
    \emph{Linux IP Masquerade HOWTO},
    \url{https://tldp.org/HOWTO/IP-Masquerade-HOWTO/index.html},
    2021.

\end{thebibliography}

\end{document}